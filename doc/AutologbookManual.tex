\documentclass[12pt,english]{article}

\usepackage{geometry}
\geometry{letterpaper, margin=1in}

\usepackage{setspace}
\onehalfspacing

\usepackage{graphicx}
\usepackage{url}
\usepackage{hyperref}
\usepackage{amssymb,amsmath}
\usepackage{subfig}
\usepackage{fancyvrb}

\newcommand{\cd}[1]{\texttt{#1}}
\newcommand{\cmd}[1]{\cd{`#1'}}
\newcommand{\cmake}{\cd{cmake}}
\newcommand{\alb}{\cd{autologbook}}


\begin{document}

\title{\alb\ lab database utilities}
\author{Michael P. Mendenhall}
\maketitle
\tableofcontents

\section{Introduction}

\alb\ is a set of utilities designed to help in common lab tasks:
	data logging and monitoring, equipment configuration, ``checklist'' routine monitoring forms.
Information is stored in \cd{sqlite3} (\url{http://www.sqlite.org}) database files,
	a highly robust and widely-used embedded database system.
Various \cd{python3} scripts facilitate reading and writing the database information,
	including a suite of \cd{cgi} scripts providing an \cd{html} web interface.

The \alb\ utilities are structured around two independent databases,
	organized for different types of information.
The ``logger'' database is intended for series of timestamped readings automatically logged
	from instruments, alongside textual annotations of lab events.
The data can be monitored and plotted through a web interface.
The ``configuration'' database is for storing configuration information for
	laboratory equipment (such as sets of voltages for power supplies),
	along with a mechanism for generating and archiving forms such as ``shift change checklist'' records.

\section{Data logger}


\section{Configurations and forms}



\end{document}

