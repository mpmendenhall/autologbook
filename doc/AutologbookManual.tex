\documentclass[12pt,english]{article}

\usepackage{geometry}
\geometry{letterpaper, margin=1in}

\usepackage{setspace}
\onehalfspacing

\usepackage{graphicx}
\usepackage{url}
\usepackage{hyperref}
\usepackage{amssymb,amsmath}
\usepackage{subfig}
\usepackage{fancyvrb}

\newcommand{\cd}[1]{\texttt{#1}}
\newcommand{\cmd}[1]{\cd{`#1'}}
\newcommand{\cmake}{\cd{cmake}}
\newcommand{\alb}{\cd{autologbook}}


\begin{document}

\title{\alb\ lab database utilities}
\author{Michael P. Mendenhall}
\maketitle
\tableofcontents

\section{Introduction}

\alb\ is a set of utilities designed to help in common lab tasks:
	data logging and monitoring, equipment configuration, ``checklist'' routine monitoring forms.
Information is stored in \cd{sqlite3} (\url{http://www.sqlite.org}) database files,
	a highly robust and widely-used embedded database system.
Various \cd{Python 3} scripts facilitate reading and writing the database information,
	including a suite of \cd{cgi} scripts providing an \cd{html} web interface.

The \alb\ utilities are structured around two independent databases,
	organized for different types of information.
The ``logger'' database is intended for series of timestamped readings automatically logged
	from instruments, alongside textual annotations of lab events.
The data can be monitored and plotted through a web interface.
The ``configuration'' database is for storing configuration information for
	laboratory equipment (such as sets of voltages for power supplies),
	along with a mechanism for generating and archiving forms such as ``shift change checklist'' records.

%
%
%
\section{Data logger}

%
%
\subsection{Database structure}
The logger database schema is defined in \cd{logger\_DB\_description.txt}.
This is organized into ``instruments'' (describing initial sources of data),
	``readouts'' (describing one particular aspect of an instrument providing a numerical readout),
	and ``readings'' (timestamped values for a readout).
An additional table of of ``log messages'' holds timestamped textual notes,
	whether automatically generated (such as run start/stop annotations from a DAQ) or hand-entered.

%
%
\subsection{Interface scripts}

%
\subsubsection{Logger server}

\cd{DB\_Logger.py} provides a local server for database interactions, providing database query and modification functions
	through an \cd{XMLRPC} interface.
Two separate server processes are launched with read-only and read/write access to the database,
	permitting database write functions to be restricted to processes on the same machine (localhost network),
	with potentially broader access for data queries.
Currently, both read and read/write servers are run on localhost only,
	until a reasonable system for broader but still restricted network access is implemented.

The logger server also provides a ``data filter'' mechanism for thinning out frequent readouts
	to a smaller number of points to be saved
	(either by simple decimation to one in every $n$,
	or recording points when (user-defined) ``significant'' changes occur).

%
\subsection{Data sources}

Data sources, capturing readings from instruments and sending them to the server,
	will need to be written for specific experimental applications.
Example ``fake'' data sources are provided in \cd{TestFunctionGen.py} and \cd{FakeHVControl.py},
	demonstrating how to set up instruments/readouts and send readings to the server process.

%
\subsection{Web monitors}



%
%
%
\section{Configurations and forms}



\end{document}

